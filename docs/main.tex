% Sablon pentru realizarea lucrarii de licenta, conform cu recomandarile
% din ghidul de redactare:
% - https://fmi.unibuc.ro/finalizare-studii/
% - https://drive.google.com/file/d/1xj9kZZgTkcKMJkMLRuoYRgLQ1O8CX0mv/view

% Multumiri lui Gabriel Majeri, acest sablon a fost creat pe baza
% codului sursa a lucrarii sale de licenta. 
% Codul sursa: https://github.com/GabrielMajeri/bachelors-thesis
% Website: https://www.gabrielmajeri.ro/
%
% Aceast sablon este licentiat sub Creative Commons Attribution 4.0 International License.

\documentclass[12pt, a4paper]{report}

% Suport pentru diacritice și alte simboluri
\usepackage{fontspec}

% Suport pentru mai multe limbi
\usepackage{polyglossia}

% Pentru linii orizontale în tabele
\usepackage{booktabs} 

% Pentru a putea folosi simboluri matematice
\usepackage{tikz}

% Setează limba textului la română
\setdefaultlanguage{romanian}
% Am nevoie de engleză pentru rezumat
\setotherlanguages{english}

% Indentează și primul paragraf al fiecărei noi secțiuni
\SetLanguageKeys{romanian}{indentfirst=true}

% Suport pentru diferite stiluri de ghilimele
\usepackage{csquotes}

\DeclareQuoteStyle{romanian}
  {\quotedblbase}
  {\textquotedblright}
  {\guillemotleft}
  {\guillemotright}

% Utilizează biblatex pentru referințe bibliografice
\usepackage[
    maxbibnames=50,
    sorting=nty
]{biblatex}

\usepackage{xpatch}  % Acest pachet ajută la modificări avansate ale stilurilor biblatex

\renewbibmacro*{issue+date}{%
  \ifboolexpr{not test {\iffieldundef{year}} or not test {\iffieldundef{issue}}}
    {\printtext[parens]{%
       \iffieldundef{issue}
         {\usebibmacro{date}}
         {\printfield{issue}%
          \setunit*{\addspace}%
          \usebibmacro{date}}}}
    {}%
  \newunit}

% Elimina "în" dacă nu există o revistă
\renewbibmacro{in:}{%
  \iffieldundef{journaltitle}
    {}
    {\bibstring{in}\intitlepunct}}


\addbibresource{bibliography.bib}

% Setează spațiere inter-linie la 1.5
\usepackage{setspace}
\onehalfspacing

% Modificarea geometriei paginii
\usepackage{geometry}

% Include funcțiile de grafică
\usepackage{graphicx}
% Încarcă imaginile din directorul `images`
\graphicspath{{./images/}}

% Listări de cod
\usepackage{listings}

% Setări pentru listările de cod
\usepackage{color}

\usepackage{tcolorbox}

\tcbuselibrary{listings, skins}

% Definirea culorilor pentru listările de cod
\definecolor{codegreen}{rgb}{0,0.6,0}
\definecolor{codegray}{rgb}{0.5,0.5,0.5}
\definecolor{codepurple}{rgb}{0.58,0,0.82}
\definecolor{backcolour}{rgb}{0.95,0.95,0.92}

\lstdefinestyle{mystyle}{
    backgroundcolor=\color{backcolour},   
    commentstyle=\color{codegreen},
    keywordstyle=\color{magenta},
    numberstyle=\tiny\color{codegray},
    stringstyle=\color{codepurple},
    basicstyle=\footnotesize,
    breakatwhitespace=false,         
    breaklines=true,                 
    captionpos=b,                    
    keepspaces=true,                 
    numbers=left,                    
    numbersep=5pt,                  
    showspaces=false,                
    showstringspaces=false,
    showtabs=false,                  
    tabsize=2
}

\lstset{style=mystyle}

\usepackage{xcolor}

\definecolor{background}{RGB}{245, 245, 245}

\definecolor{delim}{RGB}{20,105,176}
\colorlet{numb}{magenta} % Numbers
\colorlet{punct}{red} % Punctuation


% Linkuri interactive în PDF
\usepackage[
    colorlinks,
    linkcolor={black},
    menucolor={black},
    citecolor={black},
    urlcolor={blue}
]{hyperref}

% Simboluri matematice codificate Unicode
\usepackage[warnings-off={mathtools-colon,mathtools-overbracket}]{unicode-math}

% Comenzi matematice
\usepackage{amsmath}
\usepackage{mathtools}

% Formule matematice
\newcommand{\bigO}[1]{\symcal{O}\left(#1\right)}
\DeclarePairedDelimiter\abs{\lvert}{\rvert}

% Suport pentru rezumat în două limbi
% Bazat pe https://tex.stackexchange.com/a/70818
\newenvironment{abstractpage}
  {\cleardoublepage\vspace*{\fill}\thispagestyle{empty}}
  {\vfill\cleardoublepage}
\renewenvironment{abstract}[1]
  {\bigskip\selectlanguage{#1}%
   \begin{center}\bfseries\abstractname\end{center}}
  {\par\bigskip}

% Suport pentru anexe
\usepackage{appendix}

% Stiluri diferite de headere și footere
\usepackage{fancyhdr}

\usepackage{caption} % Add the caption package

\usepackage{subcaption} % Add the subcaption package

% Metadate
\title{Aplicație web pentru videoclipuri cu funcții de înțelegerea vorbirii și regăsire pe bază de text}
\author{Trifan Robert-Gabriel}

% Generează variabilele cu @
\makeatletter

\begin{document}

% Front matter
\cleardoublepage
\let\ps@plain

% Pagina de titlu
\include{0-title}
\restoregeometry
\newgeometry{
    margin=2.5cm
}

\fancypagestyle{main}{
  \fancyhf{}
  \renewcommand\headrulewidth{0pt}
  \fancyhead[C]{}
  \fancyfoot[C]{\thepage}
}

\addtocounter{page}{1}

% Rezumatul
\begin{abstractpage}

\begin{abstract}{romanian}

Popularitatea videoclipurilor a avut o creștere constantă în ultimii ani, reprezetand 82.5\% din traficul web în 2023.
Statistici recente arată că oamenii petrec în medie 17 ore pe săptămâna vizionând videoclipuri online,
acestea fiind cu 52\% mai predispuse să fie distribuite pe rețelele de socializare decât alte tipuri de conținut. \cite{bump2023video}
\par
În acest context, lucrarea de față își propune să contribuie la creșterea accesibilității și personalizării
conținutului video, prin dezvoltarea unei aplicații web care permite adăugarea de subtitrări, căutarea
videoclipurilor pe bază de text și clasificarea acestora în funcție de conținutul lor. Aplicația oferă utilizatorilor
posibilitatea de a vizualiza subtitrările în timp real și de a căuta cuvinte cheie în metadatele videoclipurilor
precum titlu, descriere, topic și subtitrare.


\end{abstract}

\begin{abstract}{english}

The popularity of videos has been steadily increasing in recent years, representing 82.5\% of web traffic in 2023.
Recent statistics show that people spend an average of 17 hours a week watching online videos, which are 52\%
more likely to be shared on social networks than other types of content. \cite{bump2023video}
\par
In this context, this work aims to contribute to increasing the accessibility and personalization of video content, by developing a web application
that allows the addition of subtitles, searching for videos based on text and classifying them according to their content. The application offers users
the possibility to view subtitles in real time and to search for keywords in the metadata of videos such as title, description, topic and subtitle.

\end{abstract}

\end{abstractpage}

\tableofcontents

% Main matter
\cleardoublepage
\pagestyle{main}
\let\ps@plain\ps@main

\chapter{Introducere}

\section{Motivație}

Motivul pentru care am ales această temă îl reprezintă nevoia de subtitrări pentru videoclipuri, în special pentru filme, 
dar și pentru tutoriale sau alte tipuri de conținut video, a căror înțelegere este îngreunată de calitatea slabă a sunetului 
sau de faptul că sunt într-o limbă străină. Astfel, în această lucrare, am ales să abordez problema subtitrărilor prin
intermediul unui model de recunoaștere a vorbirii, care extrage audio dintr-un videoclip și generează textul corespunzător.
\par
De asemenea, am ales să abordez și problema căutării videoclipurilor, care constă în căutarea cuvintelor cheie în metadate
precum titlu, descriere, topicuri sau chiar în conținutul videoclipului, folosind o bază de date special concepută pentru
acest scop, Elasticsearch. \cite{elasticsearch}
\par
Ideile menționate mai sus vor fi integrate într-o aplicație web folosind ansamblul de tehnologii MERN (MongoDB, Express.js, React.js, Node.js),
scrise în TypeScript, împreună cu 3 servere de Flask în Python pentru recunoașterea vorbirii, clasificarea videoclipurilor și
răspunderea la întrebări legate de conținutul videoclipurilor.


\section{Domenii abordate}

Această lucrare abordează 5 domenii principale:

\begin{itemize}
    \item \textbf{Procesarea semnalelor audio} - pentru extragerea audio din videoclipuri și recunoașterea vorbirii
    \item \textbf{Procesarea limbajului natural} - pentru clasificarea videoclipurilor în funcție de conținutul lor
    \item \textbf{Frontend} \textit{(React.js)} pentru interfața cu utilizatorul și pentru a oferi acces la funcționalitățile sistemului
    \item \textbf{Backend} - pentru gestionarea cererilor prin intermediul unui API, cu ajutorul a 3 servicii principale:
    \begin{itemize}
        \item \textbf{Server} \textit{(Node.js, Express.js)} pentru gestionarea cererilor și a răspunsurilor
        \item \textbf{Recunoașterea vorbirii} \textit{(Flask)} pentru gestionarea cererilor de recunoaștere a vorbirii
        \item \textbf{Clasificarea videoclipurilor} \textit{(Flask)} pentru clasificarea videoclipurilor în funcție de conținutul lor
        \item \textbf{Întrebări și răspunsuri} \textit{(Flask)} folosește API-ul de la ChatGPT pentru a răspunde la întrebări
        legate de conținutul videoclipurilor
    \end{itemize}
    \item \textbf{Baze de date} 
    \begin{itemize}
        \item \textbf{MongoDB} pentru stocarea metadatelor videoclipurilor
        \item \textbf{Elasticsearch} pentru căutarea videoclipurilor în funcție de cuvintele cheie
    \end{itemize}
\end{itemize}


\section{Structura lucrării}

Vom împărți această lucrare în 2 capitole principale:

\begin{itemize}
    \item \textbf{Concepte teoretice despre învățarea automată} - în care voi aborda \textit{recunoașterea vorbirii} și 
    \textit{clasificarea videoclipurilor} și voi prezenta setul de date folosit, arhitectura modelului,
    antrenarea și evaluarea acestuia, precum și procesările ulterioare.
    \item \textbf{Prezentarea aplicației} - în care voi aborda partea de frontend, backend și baze de date,
    precum și detaliile tehnice ale implementării.
\end{itemize}
\chapter{Inteligență artificială}

\section{Recunoașterea vorbirii}
Pentru a putea recunoaște vorbirea dintr-un videoclip, am ales să folosesc arhitectura
\textit{wav2vec 2.0} \cite{wav2vec2} dezvoltată de Facebook AI Research. Am folosit 
atât modelul \textit{facebook/wav2vec2-base-960h} antrenat pe setul de date 
\textit{LibriSpeech} \cite{librispeech}, cât și modelul preantrenat
\textit{facebook/wav2vec2-base} pe care am continuat să-l antrenez pe seturile
de date \textit{Mini LibriSpeech} (subset din LibriSpeech) și
\textit{Common Voice Delta Segment 16.1} (subset din Common Voice) \cite{commonvoice}.
\par

\subsection{Arhitectura modelului}
Modelul \textit{wav2vec 2.0} este un model de învățare profundă alcătuit din 4
componente principale: Latent Feature Encoder (Convolutional Network), Context 
Network (Transformer Encoder), Quantization Module (Gumbel Softmax) și 
Contrastive Loss. \ref{fig:wav2vec2-architecture}

\begin{figure}[h]
    \centering
    \includegraphics[width=0.8\textwidth]{wav2vec2-architecture.png}
    \caption{Arhitectura modelului \textit{wav2vec 2.0} \protect\footnotemark[1]}
    \label{fig:wav2vec2-architecture}
\end{figure}

\subsubsection{Latent Feature Encoder}
\vspace{3em}
Componenta Latent Feature Encoder este o rețea convoluțională care primește ca 
intrare un semnal audio și aplică o serie de operații de convoluție, normalizare
și activări GELU pentru a extrage caracteristici latente din semnalul audio.
\ref{fig:latent-feature-encoder}

\vspace{3em}

\begin{figure}[h]
    \centering
    \includegraphics[width=0.8\textwidth]{wav2vec2-feature-encoder.png}
    \caption{Arhitectura componentei Latent Feature Encoder \protect\footnotemark[1]}
    \label{fig:latent-feature-encoder}
\end{figure}


\vspace{3em}

\subsubsection{Context Network}
\vspace{1em}
Componenta Context Network este un encoder de tip Transformer care primește ca
intrare caracteristicile latente extrase de componenta Latent Feature Encoder și
le procesează pentru a obține o reprezentare contextuală a semnalului audio. Aducând
aminte de arhitectura modelului anterior \textit{wav2vec} \cite{wav2vec}, care folosea
tot o rețea convoluțională la acest pas, ar părea că se aseamană cu componenta anterioară.
Diferența constă în faptul că Latent Feature Encoder urmărește să reducă dimensiunea 
semnalului audio, în timp ce Context Network urmărește să înțeleagă un context mai larg
al semnalului audio. \ref{fig:wav2vec2-context-network}

\begin{figure}[h]
    \centering
    \includegraphics[width=0.8\textwidth]{wav2vec2-context-network.png}
    \caption{Arhitectura componentei Context Network \protect\footnotemark[1]}
    \label{fig:wav2vec2-context-network}
\end{figure}

\vspace{3em}

\subsubsection{Quantization Module} 
Deoarece modelul \textit{wav2vec 2.0} folosește pentru partea de Context Network un encoder de tip
Transformer, ne confruntăm cu problema structurii continue a semnalului audio. Limbajul scris poate
fi discretizat într-un set finit de simboluri, în timp ce semnalul audio nu permite în mod direct
acest lucru. Astfel, modelul \textit{wav2vec 2.0} folosește un modul de cuantizare care învață automat
unități de vorbire din semnalul audio. Intuitiv, se încearcă găsirea unor sunete fonetice 
finite și reprezentative pentru ieșirile din Latent Feature Encoder. De asemenea, se aplică 
funcția Gumbel Softmax \cite{gumbel-softmax}, funcție diferențiabilă care permite antrenarea modelului
prin backpropagation. \ref{fig:wav2vec2-quantization-module}

\begin{figure}[h]
    \centering
    \includegraphics[width=0.8\textwidth]{wav2vec2-quantization-module.png}
    \caption{Arhitectura componentei Quantization Module \protect\footnotemark[1]}
    \label{fig:wav2vec2-quantization-module}
\end{figure}

\subsubsection{Contrastive Loss}
Pentru antrenarea modelului folosește o mască care ascunde ~50\% din vectorii proiectați din spațiul
latent înainte să fie trecuți prin Context Network. Acest lucru forțează modelul să învețe
reprezentări între vectorii proiectați și vectorii ascunși. Pentru fiecare poziție mascată, se
aleg uniform aleator 100 de exemple negative de la alte poziții și se compară similaritatea cosinus
între vectorul proiectat și vectorii aleși. Astfel, funcția de pierdere contrastivă încurajează
similaritatea cu exemplele true positive și penalizează similaritatea cu exemplele false positive.

\begin{figure}[h]
    \centering 
    \includegraphics[width=0.8\textwidth]{wav2vec2-contrastive-loss.png}
    \caption{Arhitectura componentei Contrastive Loss \protect\footnotemark[1]}
    \label{fig:wav2vec2-contrastive-loss}
\end{figure}

\footnotetext[1]{Imaginile au fost preluate de pe site-ul lui Jonathan Bgn, ``Illustrated Wav2Vec 2.0'', disponibil la: \url{https://jonathanbgn.com/2021/09/30/illustrated-wav2vec-2.html}.}

\subsection{Setul de date}
Modelul oficial a fost preantrenat pe setul de date \textit{LibriSpeech}, iar eu am continuat
antrenarea pe seturile de date \textit{Mini LibriSpeech} și \textit{Common Voice Delta Segment 16.1}.

\subsubsection{Mini-LibriSpeech}
\textit{Mini LibriSpeech} este un subset al setului de date \textit{LibriSpeech} care conține 
aproximativ 2 ore de înregistrări audio la o frecvență de eșantionare de 16 kHz. În medie,
fiecare înregistrare are o durată de 6.72 secunde, cel mai lung audio având o durată de 31.5 secunde.

\subsubsection{Common Voice Delta Segment 16.1}
\textit{Common Voice Delta Segment 16.1} este un subset al setului de date \textit{Common Voice}
care conține aproximativ 2 ore de înregistrări audio la o frecvență de eșantionare de 48 kHz.
A fost nevoie să reduc frecvența de eșantionare la 16 kHz pentru a putea folosi aceste date la
antrenarea modelului. În medie, fiecare înregistrare are o durată de 5.63 secunde, cel mai lung
audio având o durată de 10.47 secunde.

\begin{figure}[h]
    \centering
    \includegraphics[width=0.8\textwidth]{length_distribution.png}
    \caption{Distribuția duratelor audio-urilor din seturile de date \textit{Mini LibriSpeech} și \textit{Common Voice Delta Segment 16.1}}
    \label{fig:length-distribution}
\end{figure}

\subsubsection{Concluzie}
Menționez aceste detalii deoarece pentru generarea subtitrărilor vom avea nevoie de audio-uri
mult mai lungi decât cele folosite pentru antrenare care nu ar încăpea în memorie. Astfel, va trebui
să folosim o tehnică de segmentare a audio-urilor în bucăți mai mici pentru a putea procesa
audio-urile mai lungi. Mai multe detalii despre această tehnică vor fi prezentate în secțiunea
\textit{Subtitări}.

\subsection{Antrenarea modelului}


\subsection{Postprocesare}
% TODO

\section{Clasificarea videoclipurilor}
% TODO

\subsection{Arhitectura modelului}
% TODO

\subsection{Setul de date}
% TODO

\subsection{Antrenarea modelului}
% TODO

\subsection{Postprocesare}
% TODO

\section{Concluzii}
% TODO

\chapter{Inginerie software}

\section{Frontend}
% TODO

\section{Backend}
% TODO
\subsection{Server}
% TODO
\subsection{Recunoașterea vorbirii}
% TODO
\subsection{Clasificarea videoclipurilor}
% TODO

\section{Baze de date}

\subsection{MongoDB}
% TODO
\subsection{Elasticsearch}
% TODO

\chapter{Concluzie}

\section{Concluzie}
Prin realizarea acestei lucrări am prezentat conceptul de recunoaștere a vorbirii dintr-un videoclip 
integrându-l într-o aplicație web relevantă. Deși scopul proiectului a fost o aplicație web,
librăria Material UI oferă posibilitatea de a fi folosită și pe dispozitivele mobile.
\par
Inițial, mi-am îndreptat atenția către modelul de recunoaștere a vorbirii și generarea subtitrărilor,
dar consider că  integrarea acestuia într-un context concret ilustrează mai bine utilitatea și
importanța acestuia. De aceea, mi-am structurat lucrarea de licență în două părți: partea de
\textbf{Inteligență Artificială}, în care am detaliat arhitectura modelului \textit{wav2vec 2.0},
cum am ales setul de date și cum am antrenat modelul, și arhitectura modelului \textit{BERT}
pentru clasificarea videoclip-urilor, explicată în aceeași manieră, folosită pentru a îmbunătăți
algoritmul de căutare, precum și partea de \textbf{Inginerie Software}, în care am prezentat arhitectura
aplicației web, tehnologiile folosite, structura frontend-ului, backend-ului, bazelor de date,
dar și a serviciilor de deployment și a serviciilor utilitare (Cron Jobs).
\par
Așa cum am menționat în secțiunea \textit{Inginerie Software}, aplicația oferă utilizatorilor
funcționalități precum: autentificare, CRUD pe informațiile proprii, dar si pe videoclip-uri și
interacțiunea cu acestea, căutarea videoclip-urilor după cuvinte cheie și suport pentru recunoașterea
vorbirii, crearea subtitrărilor și clasificarea videoclip-urilor.
\par
În procesul de dezvoltare al aplicației, am gândit întâi funcționalitățile pe care le urmăresc
și apoi am ales tehnologiile potrivite pentru a le implementa. Acest context mi-a oferit 
oportunitatea de a învăța aceste tehonologii și de a le aprofunda în lucrarea de licență.
\par
Am ales deci să utilizez stack-ul \textbf{MERN} (MongoDB, Express.js, React.js, Node.js), folosind 
limbajul TypeScript la care am  adăugat și două servere de Flask în Python. Am ales să fac partea
de antrenare a modelelor tot în Python datorită multitudinii de librării disponibile.
\par
Personal, m-am confruntat cu lipsa subtitrărilor în filme și tutoriale și consider că această
problemă nu este încă rezolvată. Cu cât domeniul de recunoaștere a vorbirii evoluează mai mult
și apar seturi de date pentru mai multe limbi, cu atât mai multe persoane vor beneficia de acces
la materiale video care nu sunt în limba lor maternă. De asemenea, consider că procesul manual
prin care omul adaugă subtitrări este unul consumator de timp și resurse, iar o soluție precum
cea prezentată în această lucrare poate simplifica munca depusă.
\par
În concluzie, lucrarea de licență prezintă un prototip de aplicație web care integrează
recunoașterea vorbirii și generarea subtitrărilor în contextul videoclip-urilor și oferă
utilizatorilor o experiență mai bună în întelegerea conținutului video.

\section{Perspective}
\par
În timpul dezvoltării am observat câteva aspecte care pot fi îmbunătățite pentru o 
performanța mai bună a aplicației. Printre acestea se numără:


\begin{itemize}
    \item \textbf{Suprimarea zgomotului}: am observat că unele videoclip-uri au melodii sau 
    zgomote de fundal care afectează performanța modelului de recunoaștere a vorbirii. Suprimarea
    acestor sunete ar putea îmbunătăți calitatea subtitrărilor.
    \item \textbf{Traducerea subtitrărilor}: m-am concentrat pe limba engleză, având mai multe
    resurse disponibile, dar consider că modelul de recunoaștere a vorbirii poate fi folosit
    și pentru alte limbi, fie prin antrenarea unui model separat, fie prin traducerea subtitrărilor
    generate.
    \item \textbf{Seturi de date mai mari}: am folosit seturi de date mici (\textit{MiniLibriSpeech},
    \textit{Common Voice Delta Segment 16.1}) din lipsa resurselor computaționale de antrenare,
    dar consider că antrenarea pe seturi de date mai mari ar îmbunătăți performanța modelului.
    \item \textbf{Topicuri mai diverse}: videoclip-urile sunt clasificate în 5 categorii
    (\textit{Tech, Sport, Business, Politics, Entertainment}), dar natura materialelor video
    este mult mai diversă de atât. O clasificare mai detaliată ar îmbunătăți experiența
    utilizatorilor.
    \item \textbf{Spell Checker}: deși am îmbunătățit modelul de recunoaștere a vorbirii
    cu n-grame, tot mai apar cuvinte greșite. Un spell checker ar putea corecta aceste greșeli.
    \item \textbf{Suport pentru subtitrări}: librăria \textit{transformers} de pe Huggingface nu oferă
    suport pentru generarea subtitrărilor și a trebuit să prelucrez separat momentele de timp
    ale cuvintelor și să le sincronizez cu videoclip-ul. În implementarea mea am ales să grupez
    câte 7 cuvinte într-o subtitrare, dar în mod normal, această valoare ar trebui să fie ajustată
    la viteza de vorbire.
    \item \textbf{Procesarea paralelă a secvențelor audio}: după cum am menționat mai sus,
    fiecare secvență audio este împărțită în bucăți de 30 de secunde, iar aceste bucăți sunt
    procesate secvențial. Nefiind dependente între ele, pot fi procesate în paralel.
    \item \textbf{Load Balancing}: fiind un prototip, am folosit un singur server pentru toate
    serviciile, dar într-un mediu de producție, arhitecturi precum \textit{nginx} sau \textit{Kubernetes}
    sunt vitale pentru existența aplicației.
    \item \textbf{Replici pentru bazele de date}: în aceeași manieră ca la punctul anterior, am folosit
    o singură bază de date MongoDB și Elasticsearch, dar în producție, arhitecturi de tip 
    \textit{Master-Slave} sau \textit{Sharding} îmbunătățesc viteza de răspuns.
    \item \textbf{Autentificare sigură}: am folosit autentificare cu token JWT, dar în producție,
    servicii de autentificare precum \textit{Auth0} sau \textit{Firebase} sunt mai sigure.
\end{itemize}

\par
Și lista poate continua. Procesul de dezvoltare adoptat a fost unul concentrat pe finalizarea
funcționalităților promise, dar intrând în detalii, sunt multe aspecte care pot fi îmbunătățite.


\printbibliography[heading=bibintoc]

\end{document}