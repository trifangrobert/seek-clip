\chapter{Concluzie}

\section{Concluzie}
Prin realizarea acestei lucrări am prezentat conceptul de recunoaștere a vorbirii dintr-un videoclip 
integrându-l într-o aplicație web relevantă. Deși scopul proiectului a fost o aplicație web,
librăria Material UI oferă posibilitatea de a fi folosită și pe dispozitivele mobile.
\par
Inițial, mi-am îndreptat atenția către modelul de recunoaștere a vorbirii și generarea subtitrărilor,
dar consider că  integrarea acestuia într-un context concret ilustrează mai bine utilitatea și
importanța acestuia. De aceea, mi-am structurat lucrarea de licență în două părți: partea de
\textbf{Concepte teoretice despre învățarea automată}, în care am detaliat arhitectura modelului \textit{wav2vec 2.0},
cum am ales setul de date și cum am antrenat modelul, și arhitectura modelului \textit{BERT}
pentru clasificarea videoclipurilor, explicată în aceeași manieră, folosită pentru a îmbunătăți
algoritmul de căutare, precum și partea de \textbf{Prezentare a aplicației}, în care am ilustrat arhitectura
aplicației web, tehnologiile folosite, structura frontend-ului, backend-ului, bazelor de date,
dar și a serviciilor de încărcare și a serviciilor utilitare (Cron Jobs).
\par
Așa cum am menționat în secțiunea \textit{Prezentarea aplicației}, aplicația oferă utilizatorilor
funcționalități precum: autentificare, CRUD pe informațiile proprii, dar si pe videoclipuri și
interacțiunea cu acestea, căutarea videoclipurilor după cuvinte cheie și suport pentru recunoașterea
vorbirii, crearea subtitrărilor și clasificarea videoclipurilor.
\par
În procesul de dezvoltare al aplicației, am gândit întâi funcționalitățile pe care le urmăresc
și apoi am ales tehnologiile potrivite pentru a le implementa. Acest context mi-a oferit 
oportunitatea de a învăța aceste tehonologii și de a le aprofunda în lucrarea de licență.
\par
Am optat pentru stack-ul \textbf{MERN} (MongoDB, Express.js, React.js, Node.js), împreună cu 
limbajul TypeScript, la care am adăugat și trei servere de Flask în Python. Am ales să fac partea
de antrenare a modelelor tot în Python datorită multitudinii de librării disponibile.
\par
Personal, m-am confruntat cu lipsa subtitrărilor în filme și tutoriale și consider că această
problemă nu este încă rezolvată. Cu cât domeniul de recunoaștere a vorbirii evoluează mai mult
și apar seturi de date pentru mai multe limbi, cu atât mai multe persoane vor beneficia de o 
experiența mai personalizața în înțelegerea conținutului video. De asemenea, consider că procesul
manual prin care omul adaugă subtitrări este unul consumator de timp și resurse, iar o soluție
precum cea prezentată în această lucrare poate simplifica munca depusă.
\par
În plus, consider că integrarea API-ului de la ChatGPT pentru răspunderea la întrebări 
legate de conținutul videoclipurilor este o funcționalitate cu mult potențial, care poate
îmbunătăți procesul de căutare și înțelegere a conținutului video.
\par
În concluzie, lucrarea de licență prezintă un prototip de aplicație web care integrează
recunoașterea vorbirii și generarea subtitrărilor în contextul videoclipurilor și oferă
utilizatorilor o experiență mai bună în întelegerea conținutului video.

\section{Perspective}
\par
În timpul dezvoltării am observat câteva aspecte care pot fi îmbunătățite atât pentru o 
performanța mai bună a aplicației, cât și pentru o utilitate mai extinsă. Printre acestea
se numără:


\begin{itemize}
    \item \textbf{Suprimarea zgomotului}: am observat că unele videoclipuri au melodii sau 
    zgomote de fundal care afectează performanța modelului de recunoaștere a vorbirii. Suprimarea
    acestor sunete ar putea îmbunătăți calitatea subtitrărilor.
    \item \textbf{Traducerea subtitrărilor}: m-am concentrat pe limba engleză, având mai multe
    resurse disponibile, dar consider că modelul de recunoaștere a vorbirii poate fi folosit
    și pentru alte limbi (fiind și țelul la care țintim cu această aplicație), fie prin antrenarea
    unui model separat, fie prin traducerea subtitrărilor generate.
    \item \textbf{Seturi de date mai mari}: am folosit seturi de date mici (\textit{MiniLibriSpeech},
    \textit{Common Voice Delta Segment 16.1}) din lipsa resurselor computaționale de antrenare,
    dar consider că antrenarea pe seturi de date mai mari ar îmbunătăți performanța modelului.
    \item \textbf{Topicuri mai diverse}: videoclipurile sunt clasificate în 5 categorii:
    (tehnologie, sport, afaceri, politică, divertisment), dar natura materialelor video este mult
    mai diversă de atât. O clasificare mai detaliată ar îmbunătăți experiența utilizatorilor.
    \item \textbf{Corector de greșeli ortografice}: deși am îmbunătățit modelul de recunoaștere a vorbirii
    cu un model de limbaj bazat pe n-grame, tot mai apar cuvinte greșite. Un corector de greșeli ortografice
    (\textit{spell checker}) ar putea îmbunătăți calitatea subtitrărilor.
    \item \textbf{Suport pentru subtitrări}: librăria \textit{transformers} de pe Huggingface nu oferă
    suport pentru generarea subtitrărilor și a trebuit să prelucrez separat momentele de timp
    ale cuvintelor și să le sincronizez cu videoclipul. În implementarea mea am ales să grupez
    câte 7 cuvinte într-o subtitrare, dar în mod normal, această valoare ar trebui să fie ajustată
    la viteza de vorbire.
    \item \textbf{Procesarea paralelă a secvențelor audio}: după cum am menționat mai sus,
    fiecare secvență audio este împărțită în bucăți de 30 de secunde, iar aceste bucăți sunt
    procesate secvențial. Nefiind dependente între ele, pot fi procesate în paralel.
    \item \textbf{Echilibrarea sarcinii de lucru (\textit{Load Balancing})}: fiind un prototip,
    am folosit un singur server pentru toate serviciile, dar într-un mediu de producție,
    arhitecturi precum \textit{nginx} sau \textit{Kubernetes} sunt vitale pentru existența aplicației.
    \item \textbf{Replici pentru bazele de date}: în aceeași manieră ca la punctul anterior, am folosit
    o singură bază de date \textit{MongoDB} și \textit{Elasticsearch}, dar în producție, arhitecturi de tip 
    \textit{Master-Slave} sau \textit{Sharding} îmbunătățesc viteza de răspuns.
    \item \textbf{Autentificare sigură}: am folosit autentificare cu \textit{token JWT}, dar în producție,
    servicii de autentificare precum \textit{Auth0} sau \textit{Firebase} sunt mai sigure.
\end{itemize}

\par
Și lista poate continua. Procesul de dezvoltare adoptat a fost unul concentrat pe finalizarea
funcționalităților promise, dar intrând în detalii, sunt multe aspecte care pot fi îmbunătățite.
