\chapter{Introducere}

\section{Motivație}

Motivul pentru care am ales această temă îl reprezintă nevoia de subtitrări pentru videoclipuri, în special pentru filme, 
dar și pentru tutoriale sau alte tipuri de conținut video, a căror înțelegere este îngreunată de calitatea slabă a sunetului 
sau de faptul că sunt într-o limbă străină. Astfel, în această lucrare, am ales să abordez problema subtitrărilor prin
intermediul unui model de recunoaștere a vorbirii, care extrage audio dintr-un videoclip și generează textul corespunzător.
\par
De asemenea, am ales să abordez și problema căutării videoclipurilor, care constă în căutarea cuvintelor cheie în metadate
precum titlu, descriere, topicuri sau chiar în conținutul videoclipului, folosind o bază de date special concepută pentru
acest scop, Elasticsearch. \cite{elasticsearch}
\par
Ideile menționate mai sus vor fi integrate într-o aplicație web folosind ansamblul de tehnologii MERN (MongoDB, Express.js, React.js, Node.js),
scrise în TypeScript, împreună cu 3 servere de Flask în Python pentru recunoașterea vorbirii, clasificarea videoclipurilor și
răspunderea la întrebări legate de conținutul videoclipurilor.


\section{Domenii abordate}

Această lucrare abordează 5 domenii principale:

\begin{itemize}
    \item \textbf{Procesarea semnalelor audio} - pentru extragerea audio din videoclipuri și recunoașterea vorbirii
    \item \textbf{Procesarea limbajului natural} - pentru clasificarea videoclipurilor în funcție de conținutul lor
    \item \textbf{Frontend} \textit{(React.js)} pentru interfața cu utilizatorul și pentru a oferi acces la funcționalitățile sistemului
    \item \textbf{Backend} - pentru gestionarea cererilor prin intermediul unui API, cu ajutorul a 3 servicii principale:
    \begin{itemize}
        \item \textbf{Server} \textit{(Node.js, Express.js)} pentru gestionarea cererilor și a răspunsurilor
        \item \textbf{Recunoașterea vorbirii} \textit{(Flask)} pentru gestionarea cererilor de recunoaștere a vorbirii
        \item \textbf{Clasificarea videoclipurilor} \textit{(Flask)} pentru clasificarea videoclipurilor în funcție de conținutul lor
        \item \textbf{Întrebări și răspunsuri} \textit{(Flask)} folosește API-ul de la ChatGPT pentru a răspunde la întrebări
        legate de conținutul videoclipurilor
    \end{itemize}
    \item \textbf{Baze de date} 
    \begin{itemize}
        \item \textbf{MongoDB} pentru stocarea metadatelor videoclipurilor
        \item \textbf{Elasticsearch} pentru căutarea videoclipurilor în funcție de cuvintele cheie
    \end{itemize}
\end{itemize}


\section{Structura lucrării}

Vom împărți această lucrare în 2 capitole principale:

\begin{itemize}
    \item \textbf{Concepte teoretice despre învățarea automată} - în care voi aborda \textit{recunoașterea vorbirii} și 
    \textit{clasificarea videoclipurilor} și voi prezenta setul de date folosit, arhitectura modelului,
    antrenarea și evaluarea acestuia, precum și procesările ulterioare.
    \item \textbf{Prezentarea aplicației} - în care voi aborda partea de frontend, backend și baze de date,
    precum și detaliile tehnice ale implementării.
\end{itemize}